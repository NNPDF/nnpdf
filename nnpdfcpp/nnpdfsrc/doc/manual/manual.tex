%% LyX 2.0.2 created this file.  For more info, see http://www.lyx.org/.
%% Do not edit unless you really know what you are doing.
\documentclass[english]{article}
\usepackage{lmodern}
\renewcommand{\ttdefault}{lmodern}
\usepackage[T1]{fontenc}
\usepackage[latin9]{inputenc}
\usepackage{listings}
\usepackage{geometry}
\geometry{verbose,tmargin=2cm,bmargin=2cm,lmargin=2cm,rmargin=2cm}
\usepackage{color}
\usepackage{float}

\makeatletter
%%%%%%%%%%%%%%%%%%%%%%%%%%%%%% User specified LaTeX commands.
\@ifundefined{definecolor}
 {\usepackage{color}}{}
\usepackage{babel}

\makeatother

\begin{document}

\title{\textbf{NNPDF++ General Overview}}


\author{S. Carrazza$^{1}$ for the NNPDF Collaboration\\
 {\small $^{1}$University of Milan}}
\maketitle
\begin{abstract}
This report presents the NNPDF++ project and objectives. It provides
the C++ code convention that must be used by every NNPDF++ developer,
in order to have a clear and optimized code, avoiding potential bugs
in the code. The rules covers two aspects of programming: the style
of code and the rules to check memory leaks before submit the changes
to subversion. We also present the new NNPDF++ structure and projects. 
\end{abstract}
\tableofcontents{}

\newpage{}


\section{Introduction}

From the development of NNPDF in Fortran we learned important concepts
about the design of computation software. Even if the Fortran language
is simple and intuitive for the development of scientific software,
today it is no longer competitive with the new object-oriented languages
like C++.


\subsection{Motivation}

The object-oriented languages provides advantages like: 
\begin{description}
\item [{\emph{(a)}}] the encapsulation of data attributes inside classes,
eliminating the global variables (common blocks) which are difficult
to track during the execution of a program, incrementing the probability
to introduce bugs and reducing the readability of code. 
\item [{\emph{(b)}}] the modularity and flexibility of code, which can
be reused easily by multiple projects, generalizing algorithms and
avoiding redundant code. 
\end{description}
On the other hand, from the point of view of performance, C++ provides: 
\begin{description}
\item [{\emph{(i)}}] fast and dynamic execution code, with C++ we have
the control of memory allocation and management. 
\item [{\emph{(ii)}}] easy interface to introduce specialized algorithms
provided by libraries such as LAPACKPP, BLAS and GSL. 
\item [{\emph{(iii)}}] easy implementation of parallel computations with
GPU and CPU technologies such as i.e. NVIDIA CUDA, OpenCL for GPU,
and OpenMP for CPU. 
\end{description}

\section{Code convention}


\subsection{Code location}

The NNPDF++ project is hosted at University of Edinburgh under revision
control. To checkout the repository do

\begin{lstlisting}
svn co https://svn.ecdf.ed.ac.uk/repo/ph/nnpdfcpp/trunk/nnpdfcpp nnpdfcpp
svn co https://svn.ecdf.ed.ac.uk/repo/ph/nnpdfcpp/trunk/data data
\end{lstlisting}



\subsection{Building the project}

To build the full project just do

\begin{lstlisting}
cd nnpdfcpp
./configure (verifies the dependencies)
\end{lstlisting}
If all dependencies are satisfied, you'll see a message explaining
the available options in the Makefile, like

\begin{lstlisting}
make install (build everything doc|gui|src)
make or make release (build src in release mode)
make debug (build src in debug mode, binary name suffix _d)
make doc (build doxygen documentation)
make qtgui (build nnpdfwizard gui program)
make uninstall (cleanup everything)
make clean (cleanup src|doc)
make cleangui (cleanup gui)
\end{lstlisting}



\subsection{Documentation}

The documentation of NNPDF++ must be compatible with Doxygen syntax.
With Doxygen the final result is very simple to read and understand,
everybody is invited to make an effort in the documentation tasks.
After doing

\begin{lstlisting}
make doc
firefox doc/doxygen/html/index.html
\end{lstlisting}
you will be able to browse the code by using your favorite web browser.


\subsection{Compiling the code}

By default we establishe\textcolor{black}{d} that the code should
be compiled using the following flags: 
\begin{itemize}
\item In the release mode, code is compiled by using


\begin{lstlisting}
CFLAGS = -Wall -O3 
\end{lstlisting}


\item However for debug mode, the compiler options are


\begin{lstlisting}
CFLAGS = -Wall -O0 -g 
\end{lstlisting}


\end{itemize}
Before the submission of code to subversion or to clusters, developers
must search for possible memory leaks in its program, by running VALGRIND
software. VALGRIND debugs code searching for memory leaks and in general
a code that passes VALGRIND test has a high level of optimization.

To use VALGRIND, we first compile the project in debug mode and then
run

\begin{lstlisting}
valgrind --leak-check=yes ./mainprogramname_d
\end{lstlisting}
which outputs the errors in the code. You are invited to solve the
issues before submit the code.

To suppress errors coming from 3th part libraries such as LHAPDF and
ROOT, you should create a suppression file following the instructions
at

\begin{lstlisting}
svn co https://svn.ecdf.ed.ac.uk/repo/ph/nnpdfcpp/branches/carrazza
\end{lstlisting}



\section{NNPDF++ structure}


\subsection{Settings}

In order to avoid the creation of multiple configuration par files
which contains configuration settings for NNPDF, we decided to create
a class which is responsible for parsing the options from a single
and unique configuration INI file. Here an example of such file syntax,
a real example is located in config/config.ini

\begin{lstlisting}
#
# Configuration file for NNPDF++,
# comments start with # or ; or [
#

[Description]
This is the description block, 
please update these lines before run.
[/Description]
################################################################
[Experiments & Datasets]
EXPERIMENT: NMCPD
	DATASET = NMCPD 0.5
EXPERIMENT: NMC
 	DATASET = NMC 0.5
EXPERIMENT: SLAC
 	DATASET = SLACP 0.5
 	DATASET = SLACD 0.5
EXPERIMENT: BCDMS
 	DATASET = BCDMSP 0.5
 	DATASET = BCDMSD 0.5
EXPERIMENT: HERA1AV
 	DATASET = HERA1NCEP 0.5
 	DATASET = HERA1NCEM 0.5
 	DATASET = HERA1CCEP 0.5
 	DATASET = HERA1CCEM 0.5
EXPERIMENT: CHORUS
 	DATASET = CHORUSNU 0.5
 	DATASET = CHORUSNB 0.5
EXPERIMENT: FLH108
 	DATASET = FLH108 1.0
EXPERIMENT: NTVDMN
 	DATASET = NTVNUDMN 0.5
 	DATASET = NTVNBDMN 0.5
EXPERIMENT: ZEUSH2
 	DATASET = Z06NC 0.5
 	DATASET = Z06CC 0.5
EXPERIMENT: ZEUSF2C
	DATASET = ZEUSF2C99 0.5
 	DATASET = ZEUSF2C03 0.5
 	DATASET = ZEUSF2C08 0.5
 	DATASET = ZEUSF2C09 0.5
EXPERIMENT: H1F2C
 	DATASET = H1F2C01 0.5
 	DATASET = H1F2C09 0.5
 	DATASET = H1F2C10 0.5
EXPERIMENT: DYE886
 	DATASET = DYE886R 1.0
	DATASET = DYE886P 0.5
EXPERIMENT: DYE605
	DATASET = DYE605 0.5
EXPERIMENT: CDFWASY
	DATASET = CDFWASYM 1.0
EXPERIMENT: CDFZRAP
	DATASET = CDFZRAP 1.0
EXPERIMENT: D0ZRAP
	DATASET = D0ZRAP 1.0
EXPERIMENT: ATLASWZRAP
	DATASET = ATLASWZRAP36PB 1.0
EXPERIMENT: D0R2CON
	DATASET = D0R2CON 0.5
EXPERIMENT: CDFR2KT
	DATASET = CDFR2KT 0.5
EXPERIMENT: ATLASR04JETS
	DATASET = ATLASR04JETS36PB 0.5
EXPERIMENT: CMSWEASY
	DATASET = CMSWEASY840PB 1.0
EXPERIMENT: LHCBWZ
	DATASET = LHCBWZ36PB 1.0
[/Experiments & Datasets]

################################################################
[Experimental Data]
T0PDFSET = NNPDF-t0-set-nlo     # PDF set to generate t0 covariance matrix
IQ2CUT   = 0			# Q2 cuts, see Filters::ApplyKinCuts() 
NPARSAT  = 2                    # number of parameters, see Filters::Q2CUTSAT()
PARSAT   = 1.5 			
	   0.333333             # array of NPARSAT elements
IREG     = 1			# Q2 cut modality
Q2MINCUT = 3.0			# Q2 minimum cut
Q2MIN    = 3.0			# Q2 minimum
W2MIN    = 12.5			# W2 minimum
[/Experimental Data]

################################################################
[ClosureTest]
FAKEDATA    = 0
FAKEPDF  = NNPDF23_nlo_as_0119
[/ClosureTest]

################################################################
[Theory]
PTORD    = 1                    # 0 for LO, 1 for NLO, 2 NNLO
ALPHAS   = 119			# Alpha_s(Mz)
Q20      = 2.0	                # the initial scale
VFNS     = GMVN                 # VFNS (FFN0, FFNS, ZMVN, GMVN)
VFNSTYPE = A                    # FONLL (A,B or C), only for GMVN
QED      = 0
NFL      = 7			# number of flavours (7, 8, 9, 13)
[/Theory]

################################################################
[Replica Properties]
SEED	     = 0                # set the seed for the random generator
GENREP       = 1		# 1 generate MC replicas, 0 use real data
RNGALGORITHM = 0		# 0 = ranlux, 1 = cmrg, see randomgenerator.cc
[/Replica Properties]

################################################################
[Fitting]
FITMETHOD  = GA                 # Minimization algorithm 
NGEN       = 50000		# Maximum number of generations
NMUTANTS   = 80			# Number of mutants for replica
NMUTPDF    = 2 2 2 2 2 2 2      # Number of mutations per PDF
MUTSIZESNG = 5    0.5
MUTPROBSNG = 0.5  0.5
MUTSIZEGLU = 5    0.5
MUTPROBGLU = 0.5  0.5
MUTSIZET3  = 5    0.5
MUTPROBT3  = 0.5  0.5
MUTSIZEV   = 5    0.5
MUTPROBV   = 0.5  0.5
MUTSIZEDS  = 5    0.5
MUTPROBDS  = 0.5  0.5
MUTSIZESP  = 5    0.5
MUTPROBSP  = 0.5  0.5
MUTSIZESM  = 5    0.5
MUTPROBSM  = 0.5  0.5
[/Fitting]

################################################################
[Stopping]
DYNSTOP    = 1 			# 1 = activate dynamic stopping
STOPMETHOD = TRVAL              # Stopping method
MINCHI2    = 6.0		# Minimum chi2 for experiments
NSMEAR     = 200		# Smear for stopping
DELTASM    = 200		# Delta smear for stopping
RV         = 1.0003		# Ratio for validation stopping
RT         = 0.9999		# Ratio for training stopping
[/Stopping]

################################################################
[Positivity]
POSMULT    = 10E10		# Positivity Lagrange Multiplier
POSDATASET = FCPOS
POSDATASET = FLPOS
POSDATASET = DMPOS
[/Positivity]

################################################################
[NN Properties]
PARAMTYPE = NN
NLAYERS   = 4			# Total number of layers
NNODES    = 2 5 3 1		# Number of nodes per layer
SMALLXSNG = 1.05 1.35		# Small x exponents for singlet 
LARGEXSNG = 2.55 3.45		# Large x exponents for singlet
SMALLXGLU = 1.05 1.35		# Small x exponents for gluon
LARGEXGLU = 3.55 4.45		# Large x exponents for gluon
SMALLXT3  = 0.00 0.50 		# Small x exponents for T3
LARGEXT3  = 2.55 3.45		# Large x exponents for T3
SMALLXV	  = 0.00 0.50		# Small x exponents for V
LARGEXV	  = 2.55 3.45		# Large x exponents for V
SMALLXDS  = -0.95 -0.65		# Small x exponents for DeltaS
LARGEXDS  = 12.0 14.0		# Large x exponents for DeltaS
SMALLXSP  = 1.05 1.35		# Small x exponents for s+
LARGEXSP  = 2.55 3.45		# Large x exponents for s+
SMALLXSM  = 0.00 0.50		# Small x exponents for s-
LARGEXSM  = 2.55 3.45		# Large x exponents for s-
[/NN Properties]

################################################################
[Ouput Folder]
RESULTSDIR = results		# Relative to main folder
[/Ouput Folder]

################################################################
[Devel]
DEBUG    = 0
[/Devel]
\end{lstlisting}

The to access the variables interactively just create an instance
of NNPDFSettings class and get the elements that you need. Where an
example

\begin{lstlisting}{[}language={C++}{]} \#include <iostream>
int main() { 
  NNPDFSettings s(``config.ini'');
  int ptOrder = s.GetPTORD();

  return 0; 
} 
\end{lstlisting}


\subsection{Projects}

The current NNPDF++ is composed by the following projects:
\begin{itemize}
\item buildmaster: which converts data provided by articles in the common data
\item validphys: which compares data results with theoretical predictions
\item filter: apply kincuts and build covmat and invcovmat (t0 and experimental for validphys and nnfit)
\item nnfit: performs the fit for 1 replica
\item chi2check: compute the chi2 for a LHgrid
\item evolcheck: debug evolution code
\item mkthprediction: print theoretical predictions
\item plotpdf: build a report with only pdf plots
\item tconvcheck: compute the convolution time for thpredictions
\item postfit: create the LHgrid after running nnfit
\item fitmanager: download and upload fits to the nnpdfserver (fit sharing) 
format
\end{itemize}


\section{Buildmaster}

The buildmaster project reads experimental data sets from articles format,
performs cross-checks on errors and creates the following files for
each experiment: 
\begin{itemize}
\item \texttt{\textbf{DATA\_<setname>.dat}} $\Rightarrow$ Contains the
experimental points + uncertainty information
\end{itemize}
This operation must be done once for all experiments and repeated
only in case new data is introduced or modified. Then, the generated
files should be located in \texttt{data/setname} folders respectively.

\subsection{Adding a new experiment}
\begin{itemize}
  \item go to data/ and create a folder with the experiment (dataset) name, you should put the data files inside a rawdata folder (look at data/CMSWEASY840PB)
  \item open include/commondata.h and add a void filter<experiment name>() method
  \item open src/commondata.cc, update the void CommonData::CreateCommonFormat()
  \item open src/commondata.cc and implement the filter<experiment name> method.
  \item to activate/deactivate the experiment you should use the configuration file.
\end{itemize}

\subsection{Running buildmaster}

Buildmaster takes as input the configuration file location. For example you can
run buildmaster as

\begin{lstlisting}
./buildmaster example.ini
\end{lstlisting}
No interaction with the filter code is needed.

\subsection{Interpreting results}

After running filter, you will find a: 
\begin{itemize}
\item \texttt{\textbf{buildmaster.log}} file in \texttt{RESULTSDIR}, which contains
the configuration used to generated filter. 
\item \texttt{\textbf{RESULTSDIR/<config name>/buildmaster}} folder containing all the generated
{*}.dat files, for each experiment set. \end{itemize}
Then you should copy the generated DATA\_<experiment>.dat file into data/<experiment> folder and upload to svn. Users should never run buildmaster.




\section{Filter}

The filter project reads experimental data sets from articles format,
performs cross-checks on errors and creates the following files for
each experiment: 
\begin{itemize}
\item \texttt{\textbf{OBS\_<setname>.dat}} $\Rightarrow$ Contains the
experimental points 
\item \texttt{\textbf{COVMAT\_<setname>.dat}}\textbf{ }$\Rightarrow$ Contains
the covariance matrix of experimental data 
\item \texttt{\textbf{INVCOVMAT\_<setname>.dat}}\textbf{ }$\Rightarrow$
Contains the inverse of the covariance matrix 
\end{itemize}
This operation must be done once for all experiments and repeated
only in case new data is introduced or modified. Then, the generated
files should be located in \texttt{data/setname} folders respectively.

\subsection{Running filter}

Filter takes as input the configuration file location. By default
it uses config/config.ini as configuration file. For example you can
run filter as

\begin{lstlisting}
./filter example.ini
\end{lstlisting}


No interaction with the filter code is needed, you just modify the
variables in the configuration file presented in Table (\ref{tab:Configuration-information.}).

\begin{table}[H]
\begin{centering}
\begin{tabular}{|c|c|c|}
\hline 
\multicolumn{1}{|c}{\texttt{\textbf{{[}data{]}}}} & \multicolumn{1}{c}{Type} & Description\tabularnewline
\hline 
\hline 
\texttt{\textbf{RESULTSDIR}} & \texttt{string} & %
\begin{minipage}[t]{0.5\columnwidth}%
The directory where output will be located.%
\end{minipage}\tabularnewline
\hline 
\hline 
\texttt{\textbf{{[}datasets{]}}} &  & \tabularnewline
\hline 
\hline 
\texttt{\textbf{DATASET}} & \texttt{string} & %
\begin{minipage}[t]{0.5\columnwidth}%
you just write line per line the name of data sets which you want
to include in the validphys computation, from Table (\ref{tab:FK-tables-status.}).%
\end{minipage}\tabularnewline
\hline
\hline
\texttt{\textbf{{[}filter{]}}} & & \tabularnewline
\hline
\hline
\texttt{\textbf{USET0}} & \texttt{int(0,1)} & %
\begin{minipage}[t]{0.5\columnwidth}%
if 0 filter will produce the experimental covariance matrix, otherwise the t0 covariance matrix will be generated.%
\end{minipage}\tabularnewline
\hline
\texttt{\textbf{T0PDFSET}} & \texttt{string} & %
\begin{minipage}[t]{0.5\columnwidth}%
PDF set to generate the t0 covariance matrix (no .LHgrid needed)%
\end{minipage}\tabularnewline
\hline
\hline
\texttt{\textbf{{[}evolution{]}}} & & \tabularnewline
\hline
\hline
\texttt{\textbf{PTORD}} & \texttt{int(1,2)} & %
\begin{minipage}[t]{0.5\columnwidth}%
the perturbation order%
\end{minipage}\tabularnewline
\hline
\texttt{\textbf{VFNS}} & \texttt{string} & %
\begin{minipage}[t]{0.5\columnwidth}%
Flavor scheme: FFN0, FFNS, ZMVN, GMVN%
\end{minipage}\tabularnewline
\hline
\texttt{\textbf{VFNSTYPE}} & \texttt{string} & %
\begin{minipage}[t]{0.5\columnwidth}%
FONLL (A, B or C), only for GMVN%
\end{minipage}\tabularnewline
\hline
\hline
\texttt{\textbf{{[}kincuts{]}}} & & \tabularnewline
\hline
\hline
\texttt{\textbf{IQ2CUT}} & \texttt{int} & %
\begin{minipage}[t]{0.5\columnwidth}%
depends on the fk table cuts.%
\end{minipage}\tabularnewline
\hline
\texttt{\textbf{NPARSAT}} & \texttt{int} & %
\begin{minipage}[t]{0.5\columnwidth}%
depends on the fk table cuts.%
\end{minipage}\tabularnewline
\hline
\texttt{\textbf{PARSAT}} & \texttt{array double} & %
\begin{minipage}[t]{0.5\columnwidth}%
depends on the fk table cuts.%
\end{minipage}\tabularnewline
\hline
\texttt{\textbf{IREG}} & \texttt{int} & %
\begin{minipage}[t]{0.5\columnwidth}%
depends on the fk table cuts.%
\end{minipage}\tabularnewline
\hline
\texttt{\textbf{Q2MINCUT}} & \texttt{double} & %
\begin{minipage}[t]{0.5\columnwidth}%
depends on the fk table cuts.%
\end{minipage}\tabularnewline
\hline
\texttt{\textbf{Q2MIN}} & \texttt{double} & %
\begin{minipage}[t]{0.5\columnwidth}%
depends on the fk table cuts.%
\end{minipage}\tabularnewline
\hline
\texttt{\textbf{W2MIN}} & \texttt{double} & %
\begin{minipage}[t]{0.5\columnwidth}%
depends on the fk table cuts.%
\end{minipage}\tabularnewline
\hline
\end{tabular}
\par\end{centering}

\caption{\label{tab:Configuration-information.}Configuration information.}
\end{table}

\subsection{Interpreting results}

After running filter, you will find a: 
\begin{itemize}
\item \texttt{\textbf{filter.log}} file in \texttt{RESULTSDIR}, which contains
the configuration used to generated filter. 
\item \texttt{\textbf{RESULTSDIR/<config name>/filter}} folder containing all the generated
{*}.dat files, for each experiment set. \end{itemize}



%\newpage{}


\section{Validphys}

Validphys produces the theoretical predictions of observables by using
the APPLgrid+NNPDF tables and the PDF set. For each experimental set,
observables are first computed and then the $\chi^{2}$ is computed
for each PDF set. Validphys is designed to work with PDFs provided
by NNPDF, CTEQ and MSTW collaborations. 


\subsection{Available FK tables}

\begin{table}[H]
\begin{centering}
{\scriptsize }%
\begin{tabular}{|c|c|l|c|c|c|c|}
\hline 
\textbf{\scriptsize Experiment} & \textbf{\scriptsize Set Type} & \textbf{\scriptsize Data Set} & \textbf{\scriptsize Filter} & \textbf{\scriptsize NLO} & \textbf{\scriptsize NNLO} & \textbf{\scriptsize Val. Status}\tabularnewline
\hline 
\hline 
\textbf{\scriptsize NMC} & {\scriptsize DIS} & \texttt{\scriptsize NMCPD} & {\scriptsize ok} & {\scriptsize ok} & {\scriptsize ok} & {\scriptsize working}\tabularnewline
\cline{2-7} 
 & {\scriptsize DIS} & \texttt{\scriptsize NMC} & {\scriptsize ok} & {\scriptsize ok} & {\scriptsize ok} & {\scriptsize working}\tabularnewline
\hline 
\textbf{\scriptsize SLAC} & {\scriptsize DIS} & \texttt{\scriptsize SLACP} & {\scriptsize ok} & {\scriptsize ok} & {\scriptsize ok} & {\scriptsize working}\tabularnewline
\cline{2-7} 
 & {\scriptsize DIS} & \texttt{\scriptsize SLACD} & {\scriptsize ok} & {\scriptsize ok} & {\scriptsize ok} & {\scriptsize working}\tabularnewline
\hline 
\textbf{\scriptsize BCDM} & {\scriptsize DIS} & \texttt{\scriptsize BCDMSP} & {\scriptsize ok} & {\scriptsize ok} & {\scriptsize ok} & {\scriptsize working}\tabularnewline
\cline{2-7} 
 & {\scriptsize DIS} & \texttt{\scriptsize BCDMSD} & {\scriptsize ok} & {\scriptsize ok} & {\scriptsize ok} & {\scriptsize working}\tabularnewline
\hline 
\textbf{\scriptsize HERA} & {\scriptsize DIS} & \texttt{\scriptsize HERA1NCEP} & {\scriptsize ok} & {\scriptsize ok} & {\scriptsize ok} & {\scriptsize working}\tabularnewline
\cline{2-7} 
 & {\scriptsize DIS} & \texttt{\scriptsize HERA1NCEM} & {\scriptsize ok} & {\scriptsize ok} & {\scriptsize ok} & {\scriptsize working}\tabularnewline
\cline{2-7} 
 & {\scriptsize DIS} & \texttt{\scriptsize HERA1CCEP} & {\scriptsize ok} & {\scriptsize ok} & {\scriptsize ok} & {\scriptsize working}\tabularnewline
\cline{2-7} 
 & {\scriptsize DIS} & \texttt{\scriptsize HERA1CCEM} & {\scriptsize ok} & {\scriptsize ok} & {\scriptsize ok} & {\scriptsize working}\tabularnewline
\hline 
\textbf{\scriptsize CHORUS} & {\scriptsize DIS} & \texttt{\scriptsize CHORUSNU} & {\scriptsize ok} & {\scriptsize ok} & {\scriptsize ok} & {\scriptsize working}\tabularnewline
\cline{2-7} 
 & {\scriptsize DIS} & \texttt{\scriptsize CHORUSNB} & {\scriptsize ok} & {\scriptsize ok} & {\scriptsize ok} & {\scriptsize working}\tabularnewline
\hline 
\textbf{\scriptsize ZEUS-H1} & {\scriptsize DIS} & \texttt{\scriptsize FLH108} & {\scriptsize ok} & {\scriptsize ok} & {\scriptsize ok} & {\scriptsize working}\tabularnewline
\hline 
\textbf{\scriptsize NuTeV} & {\scriptsize DIS} & \texttt{\scriptsize NTVNUDMN} & {\scriptsize ok} & {\scriptsize ok} & {\scriptsize ok} & {\scriptsize working}\tabularnewline
\cline{2-7} 
 & {\scriptsize DIS} & \texttt{\scriptsize NTVNBDMN} & {\scriptsize ok} & {\scriptsize ok} & {\scriptsize ok} & {\scriptsize working}\tabularnewline
\hline 
\textbf{\scriptsize ZEUS-H2} & {\scriptsize DIS} & \texttt{\scriptsize Z06NC} & {\scriptsize ok} & {\scriptsize ok} & {\scriptsize ok} & {\scriptsize working}\tabularnewline
\cline{2-7} 
 & {\scriptsize DIS} & \texttt{\scriptsize Z06CC} & {\scriptsize ok} & {\scriptsize ok} & {\scriptsize ok} & {\scriptsize working}\tabularnewline
\hline 
\textbf{\scriptsize ZEUS-F2} & {\scriptsize DIS} & \texttt{\scriptsize ZEUSF2C99} & {\scriptsize ok} & {\scriptsize ok} & {\scriptsize ok} & {\scriptsize working}\tabularnewline
\cline{2-7} 
 & {\scriptsize DIS} & \texttt{\scriptsize ZEUSF2C03} & {\scriptsize ok} & {\scriptsize ok} & {\scriptsize ok} & {\scriptsize working}\tabularnewline
\cline{2-7} 
 & {\scriptsize DIS} & \texttt{\scriptsize ZEUSF2C08} & {\scriptsize ok} & {\scriptsize ok} & {\scriptsize ok} & {\scriptsize working}\tabularnewline
\cline{2-7} 
 & {\scriptsize DIS} & \texttt{\scriptsize ZEUSF2C09} & {\scriptsize ok} & {\scriptsize ok} & {\scriptsize ok} & {\scriptsize working}\tabularnewline
\hline 
\textbf{\scriptsize H1} & {\scriptsize DIS} & \texttt{\scriptsize H1F2C01} & {\scriptsize ok} & {\scriptsize ok} & {\scriptsize ok} & {\scriptsize working}\tabularnewline
\cline{2-7} 
 & {\scriptsize DIS} & \texttt{\scriptsize H1F2C09} & {\scriptsize ok} & {\scriptsize ok} & {\scriptsize ok} & {\scriptsize working}\tabularnewline
\cline{2-7} 
 & {\scriptsize DIS} & \texttt{\scriptsize H1F2C10} & {\scriptsize ok} & {\scriptsize ok} & {\scriptsize ok} & {\scriptsize working}\tabularnewline
\hline 
\textbf{\scriptsize E605} & {\scriptsize DY} & \texttt{\scriptsize DYE605} & {\scriptsize ok} & {\scriptsize ok} & {\scriptsize ok} & {\scriptsize working}\tabularnewline
\hline 
\textbf{\scriptsize E886} & {\scriptsize DY} & \texttt{\scriptsize DYE886P} & {\scriptsize ok} & {\scriptsize ok} & {\scriptsize ok} & {\scriptsize working}\tabularnewline
\cline{3-7} 
 & {\scriptsize DY} & \texttt{\scriptsize DYE886D} & {\scriptsize ok} & {\scriptsize -} & {\scriptsize -} & {\scriptsize -}\tabularnewline
\cline{3-7} 
 & {\scriptsize DY} & \texttt{\scriptsize DYE886R} & {\scriptsize ok} & {\scriptsize ok} & {\scriptsize ok} & {\scriptsize working}\tabularnewline
\hline 
\textbf{\scriptsize CDF} & {\scriptsize WASY} & \texttt{\scriptsize CDFWASYM} & {\scriptsize ok} & {\scriptsize ok} & {\scriptsize ok} & {\scriptsize working}\tabularnewline
\cline{3-7} 
 & {\scriptsize ZRAP} & \texttt{\scriptsize CDFZRAP} & {\scriptsize ok} & {\scriptsize ok} & {\scriptsize ok} & {\scriptsize working}\tabularnewline
\cline{3-7} 
 & {\scriptsize JETS} & \texttt{\scriptsize CDFR2KT} & {\scriptsize ok} & {\scriptsize ok} & {\scriptsize ok} & {\scriptsize working}\tabularnewline
\hline 
\textbf{\scriptsize D0 } & {\scriptsize JETS} & \texttt{\scriptsize D0R2CON} & {\scriptsize ok} & {\scriptsize ok} & {\scriptsize ok} & {\scriptsize working}\tabularnewline
\cline{3-7} 
 & {\scriptsize ZRAP} & \texttt{\scriptsize D0ZRAP} & {\scriptsize ok} & {\scriptsize ok} & {\scriptsize ok} & {\scriptsize working}\tabularnewline
\hline 
\textbf{\scriptsize ATLAS } & {\scriptsize WZRAP} & \texttt{\scriptsize ATLASWZRAP36PB} & {\scriptsize ok} & {\scriptsize ok} & {\scriptsize ok} & {\scriptsize working}\tabularnewline
\cline{3-7} 
 & {\scriptsize JETS} & \texttt{\scriptsize ATLASR04JETS36PB} & {\scriptsize ok} & {\scriptsize ok} & {\scriptsize ok} & {\scriptsize working}\tabularnewline
\cline{3-7} 
 & {\scriptsize JETS} & \texttt{\scriptsize ATLASR06JETS36PB} & {\scriptsize ok} & {\scriptsize ok} & {\scriptsize ok} & {\scriptsize working}\tabularnewline
\hline 
\textbf{\scriptsize CMS } & {\scriptsize WEASY} & \texttt{\scriptsize CMSWEASY840PB} & {\scriptsize ok} & {\scriptsize ok} & {\scriptsize ok} & {\scriptsize working}\tabularnewline
\cline{2-7} 
 & {\scriptsize ZRAP} & \texttt{\scriptsize CMSZRAP} & {\scriptsize ok} & {\scriptsize -} & {\scriptsize -} & {\scriptsize -}\tabularnewline
\cline{2-7} 
 & {\scriptsize JETS} & \texttt{\scriptsize CMSINCLJETS10} & {\scriptsize ok} & {\scriptsize -} & {\scriptsize -} & {\scriptsize -}\tabularnewline
\cline{2-7} 
 & {\scriptsize JETS} & \texttt{\scriptsize CMSDIJETS10} & {\scriptsize ok} & {\scriptsize -} & {\scriptsize -} & {\scriptsize -}\tabularnewline
\hline 
\textbf{\scriptsize LHCb} & {\scriptsize ZRAP} & \texttt{\scriptsize LHCBZRAP} & {\scriptsize ok} & {\scriptsize -} & {\scriptsize -} & {\scriptsize -}\tabularnewline
\cline{2-7} 
 & {\scriptsize ZPT} & \texttt{\scriptsize LHCBZPT} & {\scriptsize ok} & {\scriptsize -} & {\scriptsize -} & {\scriptsize -}\tabularnewline
\cline{2-7} 
 & {\scriptsize MUASY} & \texttt{\scriptsize LHCBMUASY} & {\scriptsize ok} & {\scriptsize -} & {\scriptsize -} & {\scriptsize -}\tabularnewline
\cline{2-7} 
 & {\scriptsize MUCH} & \texttt{\scriptsize LHCBMUCH} & {\scriptsize ok} & {\scriptsize -} & {\scriptsize -} & {\scriptsize -}\tabularnewline
\cline{2-7} 
 & {\scriptsize WZRAP} & \texttt{\scriptsize LHCBWZ36PB} & {\scriptsize ok} & {\scriptsize ok} & {\scriptsize ok} & {\scriptsize working}\tabularnewline
\hline 
\end{tabular}
\par\end{centering}{\scriptsize \par}

\caption{\label{tab:FK-tables-status.}FK tables status. {*} Need to implement
K-factors and NP corrections. {*}{*} Asymmetries, should be tested.
{*}{*}{*} DY correction at NNLO not implemented.}
\end{table}



\subsection{Running validphys}

Validphys takes as input the configuration file address, which is
located by default in config/config.ini. For example you can run validphys
as

\begin{lstlisting}
./filter example.ini
./validphys example.ini
\end{lstlisting}


No interaction with the validphys code is needed, you just modify
the variables in the configuration file presented in Table (\ref{tab:Configuration-information.}).

\begin{table}[H]
\begin{centering}
\begin{tabular}{|c|c|c|}
\hline 
\multicolumn{1}{|c}{\texttt{\textbf{{[}data{]}}}} & \multicolumn{1}{c}{Type} & Description\tabularnewline
\hline 
\hline 
\texttt{\textbf{DATADIR}} & \texttt{string} & %
\begin{minipage}[t]{0.5\columnwidth}%
The directory where data is located.%
\end{minipage}\tabularnewline
\hline 
\texttt{\textbf{RESULSTDIR}} & \texttt{string} & %
\begin{minipage}[t]{0.5\columnwidth}%
The directory where output will be located.%
\end{minipage}\tabularnewline
\hline 
\hline 
\multicolumn{1}{|c}{\texttt{\textbf{{[}report{]}}}} & \multicolumn{1}{c}{} & \tabularnewline
\hline 
\hline 
\texttt{\textbf{PLOTFORMAT}} & \texttt{string} & %
\begin{minipage}[t]{0.5\columnwidth}%
The plot image format, all ROOT formats are supported.%
\end{minipage}\tabularnewline
\hline 
\texttt{\textbf{ERRORBAND}} & \texttt{int (0,1)} & %
\begin{minipage}[t]{0.5\columnwidth}%
PDF error band type: 1 for $1\sigma$; 0 for 68\% CL.%
\end{minipage}\tabularnewline
\hline 
\texttt{\textbf{PARFILEDIR}} & \texttt{string} & %
\begin{minipage}[t]{0.5\columnwidth}%
The directory containing the configuration {*}.par files of the current
NNPDF PDF set.%
\end{minipage}\tabularnewline
\hline 
\texttt{\textbf{PARFILESREFDIR}} & \texttt{string} & %
\begin{minipage}[t]{0.5\columnwidth}%
The directory containing the configuration {*}.par files of the reference
NNPDF set.%
\end{minipage}\tabularnewline
\hline 
\hline 
\multicolumn{1}{|c}{\texttt{\textbf{{[}validphys{]}}}} & \multicolumn{1}{c}{} & \tabularnewline
\hline 
\hline 
\texttt{\textbf{FULLREPORT}} & \texttt{int (0,1)} & %
\begin{minipage}[t]{0.5\columnwidth}%
1 for full report containing the PDFs replicas in the LH and evolution
basis; 0 for light version without those plots.%
\end{minipage}\tabularnewline
\hline 
\texttt{\textbf{PDFSET}} & \texttt{string} & %
\begin{minipage}[t]{0.5\columnwidth}%
The current NNPDF set {*}.LHgrid filename.%
\end{minipage}\tabularnewline
\hline 
\texttt{\textbf{PDFVERSION}} & \texttt{string} & %
\begin{minipage}[t]{0.5\columnwidth}%
The current pdf version number (for plots legends).%
\end{minipage}\tabularnewline
\hline 
\texttt{\textbf{PDFREFSET}} & \texttt{string} & %
\begin{minipage}[t]{0.5\columnwidth}%
The reference NNPDF set {*}.LHgrid filename.%
\end{minipage}\tabularnewline
\hline 
\texttt{\textbf{PDFREFVERSION}} & \texttt{string} & %
\begin{minipage}[t]{0.5\columnwidth}%
The reference pdf version number.%
\end{minipage}\tabularnewline
\hline 
\texttt{\textbf{PDFCTEQ}} & \texttt{string} & %
\begin{minipage}[t]{0.5\columnwidth}%
the CTEQ set {*}.LHgrid filename for comparison.%
\end{minipage}\tabularnewline
\hline 
\texttt{\textbf{PDFMSTW}} & \texttt{string} & %
\begin{minipage}[t]{0.5\columnwidth}%
The MSTW set {*}.LHgrid filename for comparison.%
\end{minipage}\tabularnewline
\hline 
\texttt{\textbf{NXPOINTS}} & \texttt{int} & %
\begin{minipage}[t]{0.5\columnwidth}%
Total number of $x$ points to plot.%
\end{minipage}\tabularnewline
\hline 
\texttt{\textbf{Q2}} & \texttt{double} & %
\begin{minipage}[t]{0.5\columnwidth}%
The energy scale for plots.%
\end{minipage}\tabularnewline
\hline 
\hline 
\texttt{\textbf{{[}PDFs{]}}} &  & \tabularnewline
\hline 
\hline 
\texttt{\textbf{NFL}} & \texttt{int} & %
\begin{minipage}[t]{0.5\columnwidth}%
Total number of parton flavors, supported values 7 and 13.%
\end{minipage}\tabularnewline
\hline 
\hline 
\texttt{\textbf{{[}evolution{]}}} &  & \tabularnewline
\hline 
\hline 
\texttt{\textbf{PTORD}} & \texttt{int} & %
\begin{minipage}[t]{0.5\columnwidth}%
The calculation pertubartive order: 0 for LO, 1 for NLO and 2 for
NNLO (used when searching for FK tables).%
\end{minipage}\tabularnewline
\hline 
\texttt{\textbf{Q20}} & \texttt{double} & %
\begin{minipage}[t]{0.5\columnwidth}%
The initial energy scale for observables.%
\end{minipage}\tabularnewline
\hline 
\hline 
\texttt{\textbf{{[}datasets{]}}} &  & \tabularnewline
\hline 
\hline 
\texttt{\textbf{DATASET}} & \texttt{string} & %
\begin{minipage}[t]{0.5\columnwidth}%
you just write line per line the name of data sets which you want
to include in the validphys computation, from Table (\ref{tab:FK-tables-status.}).%
\end{minipage}\tabularnewline
\hline 
\end{tabular}
\par\end{centering}

\caption{\label{tab:Configuration-information.}Configuration information.}
\end{table}



\subsection{Interpreting results}

After running validphys, you will find a: 
\begin{itemize}
\item \texttt{\textbf{validphys.log}} file in \texttt{REPORTDIR/<config name>}, which
contains the configuration used to generated results. 
\item \texttt{\textbf{REPORTDIR/<config name>/validphys}} folder containing

\begin{itemize}
\item \texttt{\textbf{/plots}} directory containing all the plots required
by the report in PLOTFORMAT and in {*}.root formats
\item \texttt{\textbf{/report.tex}}\textbf{ }the report generated automatically 
\item \texttt{\textbf{/Makefile}}\textbf{ }the makefile to compile and clean
report.tex, generating report.pdf 
\end{itemize}
\end{itemize}
In order to generate a pdf report, run validphys and then go to \texttt{REPORTDIR/<config name>/validphys}
folder and type make.




\section{PlotPdf}

The plotpdf project is the plotting engine which is attached to the validphys report. With this program you can generate reports with only pdf plots, such as pdf comparisons, ratios, replicas both in the evolution and flavor basis, and if the par files are available you can generate the preprocessing plots automatically.

\subsection{Running plotpdf}

For example you can run plotpdf by doing:

\begin{lstlisting}
./plotpdf example.ini plotting.ini (or simply ./plotpdf example.ini)
\end{lstlisting}

You can set custom option by editting the plotting.ini file and modifying the keys at [report], [validphys] and [PDFs] sections.

\subsection{Interpreting results}

After running plotpdf, you will find a:
\begin{itemize}
\item \texttt{\textbf{plotpdf.log}} file in \texttt{RESULTSDIR}, which contains
the configuration used to generated plotpdf. 
\item \texttt{\textbf{RESULTSDIR/plotpdf}} folder containing all the generated plots together with a makefile and report.tex, which allow you to generate a single document with all plots inside. 
\end{itemize}


%\include{nnfit}

\end{document}
